\chapter{Lektion 6}
Undersøg, om følgende rækker er konvergente eller divergente.

\begin{align*}
    \sum_{n=1}^\infty \left( \frac{2n}{n!} \right)^n , \quad
    \sum_{n=1}^\infty \frac{(-1)^n}{\sqrt{n}} \quad og \quad
    \sum_{n=1}^\infty \frac{n!}{4n^2 - 2} 
\end{align*}

\begin{bev} \textbf{1} %Nyt bevis
\newline
I dette bevis vil rodkriteriet\cite[s. 64]{"ETP"} blive brugt.
\begin{align*}
    q &= \lim_{n\to \infty} (a_n)^{\frac{1}{n}}\\
    &= \lim_{n\to \infty} ((\frac{2n}{n!})^n)^{\frac{1}{n}}\\
    &= \lim_{n\to \infty} \frac{2n}{n!}\\
    & = \lim_{n\to \infty} \frac{2}{(n-1)!}\\
\end{align*}
Af sætning 4.6 \cite[s. 49-50]{"ETP"} gælder det at
\begin{align*}
    q = \lim_{n\to \infty} \frac{2}{(n-1)!} = \frac{\lim_{n\to \infty} 2}{\lim_{n\to \infty} (n-1)!} = \frac{2}{\infty} = 0
\end{align*}
Da $q < 1$ følger det af rodkriteriet\cite[s. 64]{"ETP"} at rækken er konvergent.
\end{bev}

\begin{bev} \textbf{2} %Nyt bevis
\newline
I dette bevis vil Leibniztesten blive brugt, hvilket er:\\
Lad $a_n \geq 0$ for alle $n \in \N$, så kaldes $\sum_{n=1}^\infty (-1)^n a_n$ for en alternerende række. Hvis $\{a_n\}^\infty_{n=1}$ er aftagende og $\lim_{n\to \infty} a_n = 0$, så er dem alternerende række konvergent. Dette er bevist under lektion 6, ved en modificeret udgave af opgave 142. 

I dette tilfælde er $a_n = \frac{1}{\sqrt{n}}$, hvor der gælder at $a_n \geq 0$ for alle $n \in \N$, og dermed er $\sum_{n=1}^\infty (-1)^n \frac{1}{\sqrt{n}}$ en alternerende række. Det skal derfor vises om $a_n = \frac{1}{\sqrt{n}}$ er aftagende og om $\lim_{n\to \infty} a_n  =\lim_{n\to \infty}\frac{1}{\sqrt{n}} = 0$.

Da kvadratrods-funktionen er defineret som en voksende funktion må der gælde at 
\begin{align*}
    \sqrt{n}<\sqrt{n+1}
\end{align*}
Og heraf gælder det for $a_n$ at
\begin{align*}
    \frac{1}{\sqrt{n}} > \frac{1}{\sqrt{n+1}}
\end{align*}
Dermed vides det at $a_n$ er en aftagende funktion. Nu skal grænseværdien for $a_n$ bestemmes, hvilket gøres ved brug af sætning 4.6 \cite[s. 49-50]{"ETP"}.
\begin{align*}
    \lim_{n\to \infty} a_n  &=\lim_{n\to \infty}\frac{1}{\sqrt{n}}\\
    &=\frac{lim_{n\to \infty}1}{lim_{n\to \infty}\sqrt{n}}\\
    & = \frac{1}{\infty} = 0
\end{align*}
Altså er $\lim_{n\to \infty} a_n = 0$. 

Da $\sum_{n=1}^\infty \frac{(-1)^n}{\sqrt{n}}$ er en alternerende række, og $a_n$ både er aftagende og har grænseværdien nul. Er rækken konvergent.
\end{bev}

\begin{bev} \textbf{3} %Nyt bevis
\newline
I dette bevis vil kvotientkriteriet\cite[s. 63]{"ETP"} blive brugt. Hvor $a_n = \frac{n!}{4n^2 - 2}$
\begin{align*}
    q &= \lim_{n\to \infty}\frac{a_{n+1}}{a_n}\\
    &= \lim_{n\to \infty}\frac{\frac{(n+1)!}{4(n+1)^2 - 2} }{\frac{n!}{4n^2 - 2}}
\end{align*}
Dette vil nu blive reduceret 
\begin{align*}
    q &= \lim_{n\to \infty}\frac{(n+1)!(4n^2-2)}{(4(n+1)^2-2)n!}\\
    &= \lim_{n\to \infty}\frac{(n+1)(4n^2-2)}{4n^2+8n+2}\\
    &= \lim_{n\to \infty}\frac{(n+1)(2n^2-1)}{2n^2+4n+1}\\
    &= \lim_{n\to \infty}\frac{2n^3+2n^2-n-1}{2n^2+4n+1}\\
    &= \lim_{n\to \infty}\frac{n^2(2n+2-\frac{1}{n}-\frac{1}{n^2})}{n^2(2+\frac{4}{n}+\frac{1}{n^2})}\\
    &= \lim_{n\to \infty}\frac{2n+2-\frac{1}{n}-\frac{1}{n^2}}{2+\frac{4}{n}+\frac{1}{n^2}}
\end{align*}
Nu bruges sætning 4.6 \cite[s. 49-50]{"ETP"} 
\begin{align*}
    q = \lim_{n\to \infty}\frac{2n+2-\frac{1}{n}-\frac{1}{n^2}}{2+\frac{4}{n}+\frac{1}{n^2}} = \frac{\lim_{n\to \infty}2n+2-\frac{1}{n}-\frac{1}{n^2}}{\lim_{n\to \infty}2+\frac{4}{n}+\frac{1}{n^2}} = \frac{\infty}{2} > 1
\end{align*}
Da $q>1$ følger det af sætning 4.38\cite[s. 63]{"ETP"} at rækken er divergent. 
\end{bev}