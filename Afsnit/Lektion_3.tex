\chapter{Lektion 3}

Lad $\{a_n\}^\infty_{n=1}$, $\{b_n\}^\infty_{n=1}$ og $\{x_n\}^\infty_{n=1}$ være reelle talfølger således, at $x_n$ er "klemt inde" mellem $a_n$ og $b_n$, dvs.
%
\begin{align*}
    a_n \leq x_n \leq b_n \quad \text{for alle} \quad n\in \N
\end{align*}

Bevis, at hvis talfølgerne $\{a_n\}^\infty_{n=1}$ og $\{b_n\}^\infty_{n=1}$ er konvergente med den samme grænseværdi $c$, så er $\{x_n\}^\infty_{n=1}$ også konvergent med grænseværdien $c$.

\subsubsection{Besvarelse}
Det vides at
\begin{align}
a_n &\leq x_n \leq b_n \nonumber \\
\lim_{n\rightarrow \infty} a_n &= \lim_{n\rightarrow \infty} b_n = c \label{lektion3_1}
\end{align}

\begin{bev} \textbf{} %Nyt bevis
\newline
Ud fra ovenstående fås
\begin{align*}
    a_n - c \leq x_n - c \leq b_n - c
\end{align*}
Først bestemmes grænserne for $x_n-c$. 
\begin{align*}
    x_n - c &\leq b_n - c\\
    -(x_n - c) &\leq -(a_n - c)
\end{align*}
Ud fra dette haves at 
\begin{align*}
    |x_n - c| \leq max\{b_n-c, -(a_n - c)\} \leq max\{|b_n-c|, |a_n - c|\}
\end{align*}
$|b_n-c|$ kaldes $N_b$ og $|a_n - c|$ kaldes $N_a$. Disse går mod 0, hvilket ses ud fra \autoref{lektion3_1}. Derfor kan man altid finde en værdi med dem, som er under $\epsilon$. Givet $\epsilon>0$, så lad $N=max\{N_a,N_b\}$ afparere for $\{a_n\}^\infty_{n=1}$ og $\{b_n\}^\infty_{n=1}$. Så gælder det at
\begin{align*}
    n\geq N \Rightarrow |b_n-c|<\epsilon \text{og} |a_n-c|<\epsilon \Rightarrow |x_n - c| < \epsilon
\end{align*}
Derudfra gælder det altså at $x=c$, altså at $\{x_n\}^\infty_{n=1}$ også er konvergent med grænseværdien $c$.
\end{bev}